% This is the template used for the GIH thesis. It is based on 
% Reed College LaTeX thesis template. Most of the work (Reed template)
% for the document class was done by Sam Noble (SN), as well as this
% template. Later comments etc. by Ben Salzberg (BTS). Additional
% restructuring and APA support by Jess Youngberg (JY).
%
% See http://web.reed.edu/cis/help/latex.html for help. There are a
% great bunch of help pages there, with notes on
% getting started, bibtex, etc. Go there and read it if you're not
% already familiar with LaTeX.
%
% Any line that starts with a percent symbol is a comment.
% They won't show up in the document, and are useful for notes
% to yourself and explaining commands.
% Commenting also removes a line from the document;
% very handy for troubleshooting problems. -BTS

% The template was updated by Daniel Hammarström to fit GIH
% requirements. Additional code was borrowed from the 
% Stockholm University (Andreas Solders 2011) template.

% The template was forked from the thesisdown package (CII updates)

%%
%% Preamble
%%
% \documentclass{<something>} must begin each LaTeX document
\documentclass[twoside,10pt]{gihclass} %Default style using S5 paper
% Packages are extensions to the basic LaTeX functions. Whatever you
% want to typeset, there is probably a package out there for it.
% Chemistry (chemtex), screenplays, you name it.
% Check out CTAN to see: http://www.ctan.org/
%%
\usepackage{graphicx,latexsym}
\usepackage{amsmath}
\usepackage{amssymb,amsthm}
\usepackage{longtable,booktabs,setspace}
\usepackage{chemarr} %% Useful for one reaction arrow, useless if you're not a chem major
\usepackage[hyphens]{url}
% Added by CII
\usepackage{hyperref,xcolor}
\hypersetup{
    colorlinks = false,
    pdfborder={0 0 0}
}
\usepackage{lmodern}
\usepackage{float}
\floatplacement{figure}{H}
% End of CII addition
\usepackage{rotating}

% Next line commented out by CII
%%% \usepackage{natbib}
% Comment out the natbib line above and uncomment the following two lines to use the new
% biblatex-chicago style, for Chicago A. Also make some changes at the end where the
% bibliography is included.
%\usepackage{biblatex-chicago}
%\bibliography{thesis}


% Added by CII (Thanks, Hadley!)
% Use ref for internal links
\renewcommand{\hyperref}[2][???]{\autoref{#1}}
\def\chapterautorefname{Chapter}
\def\sectionautorefname{Section}
\def\subsectionautorefname{Subsection}
% End of CII addition

% Added by CII
\usepackage{caption}
\captionsetup{width=5in}
% End of CII addition


% \usepackage{times} % other fonts are available like times, bookman, charter, palatino


% Syntax highlighting #22

% To pass between YAML and LaTeX the dollar signs are added by CII

% New variables 2019-02-06 GIH copyright info 
\isbn{Provided by the library} 
\place{Stockholm}
\printeby{Printer service, Stockholm, 2019}
\coverinfo{}
\year{2019}



\title{Determinants of intra-individual variation in adaptability to resistance training of different volumes with special reference to skeletal muscle phenotypes}
\author{Daniel Hammarström}
% The month and year that you submit your FINAL draft TO THE LIBRARY (May or December)
\date{May 20xx}


%If you have two advisors for some reason, you can use the following
% Uncommented out by CII
\sernr{999}
% End of CII addition
%%% Remember to use the correct department!

% if you're writing a thesis in an interdisciplinary major,
% uncomment the line below and change the text as appropriate.
% check the Senior Handbook if unsure.
%\thedivisionof{The Established Interdisciplinary Committee for}
% if you want the approval page to say "Approved for the Committee",
% uncomment the next line
%\approvedforthe{Committee}

% Added by CII
%%% Copied from knitr
%% maxwidth is the original width if it's less than linewidth
%% otherwise use linewidth (to make sure the graphics do not exceed the margin)
\makeatletter
\def\maxwidth{ %
  \ifdim\Gin@nat@width>\linewidth
    \linewidth
  \else
    \Gin@nat@width
  \fi
}
\makeatother

\renewcommand{\contentsname}{Table of Contents}
% End of CII addition

\setlength{\parskip}{0pt}

% Added by CII

\providecommand{\tightlist}{%
  \setlength{\itemsep}{0pt}\setlength{\parskip}{0pt}}




\Dedication{
You can have a dedication here if you wish.
}

\Preface{

}

\Abstract{
The preface pretty much says it all.

\par

Second paragraph of abstract starts here.
}

\Listofpapers{
\begin{enumerate}
\def\labelenumi{\Roman{enumi}.}
\item
  \textbf{Hammarström D}, Øfsteng S, Koll L, Hanestadhaugen M, Hollan I, Apró W, Blomstrand E, Rønnestad B, Ellefsen S Benefits of higher resistance-training volume are related to ribosome biogenesis. The \emph{Journal of physiology}. 2020;598(3):543-65.
\item
  Khan Y, \textbf{Hammarström D}, Rønnestad B, Ellefsen S, Ahmad R Increased biological relevance of transcriptome analyses in human skeletal muscle using a model-specific pipeline. \emph{Submitted.}
\item
  \textbf{Hammarström D}, Øfsteng S, Jacobsen N, Flobergseter K, Rønnestad B, Ellefsen S Ribosome accumulation during early phase resistance training. \emph{Manuscript}
\end{enumerate}
}


	\usepackage{lettrine} \usepackage{booktabs} \usepackage{longtable} \usepackage{array} \usepackage{multirow} \usepackage{wrapfig} \usepackage{float} \usepackage{colortbl} \usepackage{pdflscape} \usepackage{tabu} \usepackage{threeparttable} \usepackage{threeparttablex} \usepackage[normalem]{ulem} \usepackage{makecell} \usepackage{siunitx}
% End of CII addition
%%
%% End Preamble
%%
%


% Added updated related to csl update in Pandoc
\newlength{\cslhangindent}
\setlength{\cslhangindent}{1.5em}
\newlength{\csllabelwidth}
\setlength{\csllabelwidth}{3em}
\newenvironment{CSLReferences}[3] % #1 hanging-ident, #2 entry spacing
 {% don't indent paragraphs
  \setlength{\parindent}{0pt}
  % turn on hanging indent if param 1 is 1
  \ifodd #1 \everypar{\setlength{\hangindent}{\cslhangindent}}\ignorespaces\fi
  % set entry spacing
  \ifnum #2 > 0
  \setlength{\parskip}{#2\baselineskip}
  \fi
 }%
 {}
\usepackage{calc} % for \widthof, \maxof
\newcommand{\CSLBlock}[1]{#1\hfill\break}
\newcommand{\CSLLeftMargin}[1]{\parbox[t]{\maxof{\widthof{#1}}{\csllabelwidth}}{#1}}
\newcommand{\CSLRightInline}[1]{\parbox[t]{\linewidth}{#1}}
\newcommand{\CSLIndent}[1]{\hspace{\cslhangindent}#1}


\begin{document}




% Everything below added by CII


\frontmatter % this stuff will be roman-numbered
% \pagestyle{empty} % this removes page numbers from the frontmatter
  \maketitle
  \begin{dedication}
  \topskip0pt
\vspace*{\fill}
 You can have a dedication here if you wish.
\vspace*{\fill}
  \end{dedication}
\begin{defence}
    THESIS FOR DOCTORAL DEGREE (Ph.D.)\\
    ~\\
    ~\\
    \textbf{The title of your thesis}\\
    ~\\
    by\\
    \textbf{Your name}\\
    ~\\
    ~\\
    Thesis for Philosophy of Doctoral Degree in Sport Sciences, at The Swedish School of Sport and Health Sciences (GIH), which, according to the decision of the dean, will be publicly defended on \emph{DATE}. The thesis defense will be held at the auditorium at The Swedish School of Sport and Health Sciences (GIH), Stockholm.\\
    ~\\
    ~\\
    \textbf{Opponent}\\
    Profesor \ldots.\\
    ~\\
    \textbf{Principal supervisor}\\
    Profesor\ldots{}\\
    ~\\
    \textbf{Co-supervisor(s)}\\
    -Professor\ldots{}\\
    -Professor\ldots{}\\
    -Professor\ldots{}\\
    ~\\
    \textbf{Examination board}\\
    -Associate professor\ldots{}\\
    -Professor \ldots{}\\
    -Professor \ldots{}
  \end{defence}

  \begin{abstract}
    The preface pretty much says it all.

    \par

    Second paragraph of abstract starts here.
  \end{abstract}
  \begin{listofpapers}
    \begin{enumerate}
    \def\labelenumi{\Roman{enumi}.}
    \item
      \textbf{Hammarström D}, Øfsteng S, Koll L, Hanestadhaugen M, Hollan I, Apró W, Blomstrand E, Rønnestad B, Ellefsen S Benefits of higher resistance-training volume are related to ribosome biogenesis. The \emph{Journal of physiology}. 2020;598(3):543-65.
    \item
      Khan Y, \textbf{Hammarström D}, Rønnestad B, Ellefsen S, Ahmad R Increased biological relevance of transcriptome analyses in human skeletal muscle using a model-specific pipeline. \emph{Submitted.}
    \item
      \textbf{Hammarström D}, Øfsteng S, Jacobsen N, Flobergseter K, Rønnestad B, Ellefsen S Ribosome accumulation during early phase resistance training. \emph{Manuscript}
    \end{enumerate}
  \end{listofpapers}

  \hypersetup{linkcolor=black}
  \setcounter{tocdepth}{2}
  \tableofcontents

  \listoftables

  \listoffigures




\mainmatter % here the regular arabic numbering starts
\pagestyle{fancyplain} % turns page numbering back on

\hypertarget{thesisdownthesis_gitbook-default}{%
\chapter{thesisdown::thesis\_gitbook: default}\label{thesisdownthesis_gitbook-default}}

Placeholder

\hypertarget{background}{%
\chapter{Background}\label{background}}

\hypertarget{exercise-prescription}{%
\section{Exercise prescription}\label{exercise-prescription}}

Systematic physical exercise with the purpose to improve health or physical performance has been shown to be part of many early civilizations
(\textbf{RN2640?})
Today's exercise-training prescription still bears traces of ideas from these eras, further developed during the renaissance and formalized in systems like Ling gymnastics during the nineteenth century
(1)
which in turn is referenced in twentieth century texts on exercise prescription
(2).
With the introduction of heavy-resistance exercises for the development of muscle strength and mass after injury, DeLorme outlined a system on which modern resistance-training (RT) exercise prescription is based (3).
In contrast to previous recommendation
(2),
DeLorme specifically emphasized high-resistance, low-repetition exercise where progression was achieved with increased resistance as opposed to endurance-like exercise where progression was achieved through increased number of repetitions
(2).
The concept of repetition maximum (RM) as a way of prescribing and individual dosage and monitoring progress was introduced from case reports (3) and verified as a efficient way of improwing muscular strength
(4).
DeLorme originally prescribed sessions of up to 100 repetitions performed in sets of 10 (3) but later revised this recommendation to three sets of 10 repetitions performed with increasing intensities
(\textbf{RN2641?}, 5).

\hypertarget{adaptations-to-resistance-training}{%
\section{Adaptations to resistance training}\label{adaptations-to-resistance-training}}

\hypertarget{muscle-hypertrophy-and-strength}{%
\subsection{Muscle hypertrophy and strength}\label{muscle-hypertrophy-and-strength}}

A well characterized response to systematic resistance training in humans is muscle growth. On the whole muscle level, resistance training can be expected to result in increases of 6-9\%

(6)

On the muscle fiber level

\hypertarget{muscle-fiber-type-transitions}{%
\subsection{Muscle fiber-type transitions}\label{muscle-fiber-type-transitions}}

\hypertarget{mitochondrial-function}{%
\subsection{Mitochondrial function}\label{mitochondrial-function}}

Increased mitochondrial respiration after 12 weeks of RT in young men (7)

Fiber type distributions do not predict muscle mitochondrial density in endurance trained individuals (8)

\url{PMID:158694} Reduced mitochondrial density per fiber area in reponse to RT

\hypertarget{effects-of-exercise-prescription-on-muscle-mass-and-strength}{%
\section{Effects of exercise prescription on muscle mass and strength}\label{effects-of-exercise-prescription-on-muscle-mass-and-strength}}

Precise exercise-training\footnote{Exercise is herein defined as an acute bout of physical activity designed to affect physical characteristics such as strength, speed or endurance. Training is defined as the systematic process of combining multiple exercise-sessions performed in sequence over time. Resistance-exercise is defined as an acute strength-promoting program requiring the neuromuscular system to exert force against resistance. Resistance training is defined as a long-term process of multiple resistance exercise-sessions performed over a defined period of time.}
prescription gives information on exercises, their sequential order, intensity and volume, rest periods between efforts or sessions and the frequency at which exercise sessions are to be performed
(10).
By manipulating these variables, resistance training programs can be tailored to better fit goals and starting points of any individual.
The relative importance of exercise-training variables for training outcomes has been examined in numerous studies including (but not limited to) the overall organization of exercise sessions,
(12, 14)
training frequency
(16),
and intensity
(18).
It could be argued that training volume is of particular importance for muscle growth as when this variable is held constant, manipulation of other variables has little or no effect hypertrophy
(20, 18).
For development of strength, factors such as intensity and within session organization of exercises is of importance
(22, 24),
however, when other factors are held constant, increased training volume generally leads to increased strength
(22,26, 28),
similarly to effects of training volume on muscle growth
(10,30).

\hypertarget{effects-of-resistance-exercise-volume-on-muscle-strength-and-mass}{%
\subsection{Effects of resistance exercise volume on muscle strength and mass}\label{effects-of-resistance-exercise-volume-on-muscle-strength-and-mass}}

Exercise volume can be prescribed as the within session number of sets performed per muscle group. This unit is practical as it comparable between individuals and muscle groups (32).
Berger conducted an early study concerning effects of resistance exercise volume with the goal to determine what method most efficiently produced strength gains (in healthy young males) (34). Berger compared one, two and three sets performed with two, six or ten repetition maximum (RM) in the bench press, three times per week, over twelve weeks. As the combined effect of three sets per session was superior regardless of the number of repetitions performed Berger concluded in favor of three sets. This conclusion was later challenged on the basis of data interpretation
(36, 38).
Reveiwing the study by Berger and others, Carpinelli and Otto arrived to the conclusion that there was ``insufficient evidence to support the prevalent belief that a greater volume of exercise (through multiple sets) will elicit superior muscular strength or hypertrophy'' (36). This stand has since been repeatedly put forward as a criticism of higher volume training programs
(39,41) and sparked considerable scientific activity. The main argument against the recommendation of additional volume in strength training programs has been the lack of statistically significant results in single studies (38,39).
Indeed, individual studies do not generally agree on dose-dependent effects of training volume on muscle mass and strength gains
(43, 45, 47, 49, 51, 53, 55, 57, 59, 61, 63, 65),
including studies performed within participants, where different training volumes are allocated to either extremity
(67, 69).
For example, differences in strength are between volume conditions are found in older individuals
(43, 45, 55)
but not confirmed in another study
(51){]}.
Studies shows that more volume does not lead to increased muscle gains in young individuals
(61, 57, 47)
a conclusion challenged by others
(65, 49).

As previously noted, combining the above results and additional studies, meta-analyses concluded that training volume dose-dependency exists for the development of muscle mass and strength
{[}(22);
(26);
(28);
(10,30).
As a second argument against additional volume in resistance training recommendation has been the cost/benefit relationship of adding training volume without meaningful or substantial additional gains
(38, 39),
a subsequent question is, whom would benefit from greater volumes and whom would not?
Schoenfeld \emph{et al.} combined data from published studies to explore if participant characteristics of the above mentioned studies interacted with training volume in explaining study outcomes. Neither sex, muscle groups nor age interacted with volume prescription indicating that no such factor would be able refine training prescription guidelines
(30).
As the number of studies used to synthesis the meta-analysis was relatively low (\emph{n} = 15) and the studies were heterogeneous in terms of e.g.~outcome measurements, it may have lacked in power to detect any meaningful interactions. Additionally, included studies may not have been reporting relevant characteristics for such analysis.

Collectively, the available evidence suggest that there is overlap between training outcomes in studies were different volume has been utilized.
The overlap cannot, with available data, be explained by general population characteristics such as age or sex.
Studying the effect of different training volumes within participants could potentially help to define determinants of training outcomes in response to different volume conditions.
Two within-participant studies have investigated the effects of training volume on strength and hypertrophy outcomes.
Sooneste \emph{et al.} compared strength outcomes in response to three- and one-set elbow flexor training for 12 weeks in young males using a whitin-participant protocol (arms allocated to either volume condition).
The results showed general benefit of three- over one-set training for muscle hypertrophy and tended to do so also for strength gains (69).
No attempts were made to relate baseline characteristics to the magnitude of differences between volume conditions, presumably due to the small sample size (\emph{n} = 8).
Mitchell \emph{et al.} compared muscle hypertrophy and strength gains in response to three- and one-set of knee-extension exercise performed three times per week for ten weeks.
The study contained an additional training condition (low intensity, 30\% of 1RM performed with three sets) with participants legs assigned to either of the three conditions in a random fashion.
No significant differences were reported between volume conditions for muscle mass or strength gains (67).
However, the analyses were performed without taking the correlation between individuals into account due to the mixed design (67).
No attempts were made to relate any measured characteristic to differences in responses.

\hypertarget{molecular-determinants-of-training-induced-muscle-hypertrophy}{%
\section{Molecular determinants of training-induced muscle hypertrophy}\label{molecular-determinants-of-training-induced-muscle-hypertrophy}}

Muscle mass change as a consequence of muscle protein synthesis and breakdown. When a net positive balance is achieved the muscle increase in mass. Resistance exercise leads ta acute blunting of muscle protein synthesis followed by an increase over resting levels in the post exercise period
.
.

\hypertarget{ribosomal-biogenesis}{%
\subsection{Ribosomal biogenesis}\label{ribosomal-biogenesis}}

(71)
(72)

(73)

\hypertarget{transcriptional-regulation-of-muscle-mass}{%
\subsection{Transcriptional regulation of muscle mass}\label{transcriptional-regulation-of-muscle-mass}}

(74)

(75)

.

.

\hypertarget{aims}{%
\chapter{Aims}\label{aims}}

The primary aim of this thesis was to relate the adaptive response to resistance training with low- and moderate-volume to skeletal-muscle characteristics in previously untrained individuals. The key question was whether manipulation of exercise-volume will have diverse effects in different individuals related to muscular intrinsic characteristics. A further aim was to characterize exercise-volume dependence in muscle molecular characteristics and determine a time course profile of markers of ribosomal biogenesis in response to resistance training. Based on these aims, the objectives of the present thesis were;
\begin{itemize}
\tightlist
\item
  to relate skeletal muscle and systemic characteristics to benefit of moderate- compared to low-volume resistance training;
\item
  To determine volume-dependence in molecular networks related to muscle growth and remodeling in response to resistance training
\item
  To determine a time course of markers related to ribosome biogenesis in the early phase of resistance training.
\end{itemize}
\hypertarget{methods}{%
\chapter{Methods}\label{methods}}

Placeholder

\hypertarget{study-participants-protocols-and-training-interventions}{%
\section{Study participants, protocols and training interventions}\label{study-participants-protocols-and-training-interventions}}

\hypertarget{resistance-training-interventions}{%
\section{Resistance training interventions}\label{resistance-training-interventions}}

\hypertarget{ethical-considerations}{%
\subsection{Ethical considerations}\label{ethical-considerations}}

\hypertarget{gene-expression-analysis}{%
\section{Gene expression analysis}\label{gene-expression-analysis}}

\hypertarget{determination-of-protein-abundance}{%
\section{Determination of protein abundance}\label{determination-of-protein-abundance}}

\hypertarget{statistics-and-data-analysis}{%
\section{Statistics and data analysis}\label{statistics-and-data-analysis}}

\hypertarget{gene-expression-analysis-1}{%
\section{Gene expression analysis}\label{gene-expression-analysis-1}}

\hypertarget{normalization}{%
\subsection{Normalization}\label{normalization}}

\hypertarget{literature-search-inclusion-criteria-and-coding-of-studies}{%
\subsection{Literature search, inclusion criteria and coding of studies}\label{literature-search-inclusion-criteria-and-coding-of-studies}}

\hypertarget{calculations-of-effect-sizes-and-statistical-analysis}{%
\subsection{Calculations of effect sizes and statistical analysis}\label{calculations-of-effect-sizes-and-statistical-analysis}}

\hypertarget{results-and-discussion}{%
\chapter{Results and Discussion}\label{results-and-discussion}}

Placeholder

\hypertarget{effects-of-different-training-volume-on-changes-in-muscle-size-and-function}{%
\section{Effects of different training volume on changes in muscle size and function}\label{effects-of-different-training-volume-on-changes-in-muscle-size-and-function}}

\hypertarget{acute-effects-of-diffrent-training-volume-on-determinants-of-muscle-protein-synthesis}{%
\section{Acute effects of diffrent training volume on determinants of muscle protein synthesis}\label{acute-effects-of-diffrent-training-volume-on-determinants-of-muscle-protein-synthesis}}

\hypertarget{general-discussion}{%
\chapter{General Discussion}\label{general-discussion}}

\hypertarget{conclusion}{%
\chapter*{Conclusion}\label{conclusion}}
\addcontentsline{toc}{chapter}{Conclusion}

If we don't want Conclusion to have a chapter number next to it, we can add the \texttt{\{-\}} attribute.

\textbf{More info}

And here's some other random info: the first paragraph after a chapter title or section head \emph{shouldn't be} indented, because indents are to tell the reader that you're starting a new paragraph. Since that's obvious after a chapter or section title, proper typesetting doesn't add an indent there.

\hypertarget{references}{%
\chapter*{References}\label{references}}
\addcontentsline{toc}{chapter}{References}

Placeholder

\hypertarget{refs}{}
\begin{CSLReferences}{0}{0}
\leavevmode\hypertarget{ref-RN2636}{}%
\CSLLeftMargin{1. }
\CSLRightInline{Ling PH. Gymnastikens allmänna grunder {[}elektronisk resurs{]} {[}Internet{]}. Upsala: Palmblad \& Comp.; 1834. }

\leavevmode\hypertarget{ref-RN2634}{}%
\CSLLeftMargin{2. }
\CSLRightInline{Nicoll EA. Principles of exercise therapy. British medical journal {[}Internet{]}. 1943;1(4302):747--50. }

\leavevmode\hypertarget{ref-RN2633}{}%
\CSLLeftMargin{3. }
\CSLRightInline{Delorme TL. RESTORATION OF MUSCLE POWER BY HEAVY-RESISTANCE EXERCISES. JBJS {[}Internet{]}. 1945;27(4). }

\leavevmode\hypertarget{ref-RN2632}{}%
\CSLLeftMargin{4. }
\CSLRightInline{Houtz SJ, Parrish AM, Hellebrandt FA. The influence of heavy resistance exercise on strength. Physical Therapy {[}Internet{]}. 1946;26(6):299--304. }

\leavevmode\hypertarget{ref-RN2639}{}%
\CSLLeftMargin{5. }
\CSLRightInline{Todd JS, Shurley JP, Todd TC. Thomas l. DeLorme and the science of progressive resistance exercise. J Strength Cond Res. 2012;26(11):2913--23. }

\leavevmode\hypertarget{ref-RN2629}{}%
\CSLLeftMargin{6. }
\CSLRightInline{Ikai M, Fukunaga T. A study on training effect on strength per unit cross-sectional area of muscle by means of ultrasonic measurement. Int Z Angew Physiol {[}Internet{]}. 1970;28(3):173--80. }

\leavevmode\hypertarget{ref-RN2608}{}%
\CSLLeftMargin{7. }
\CSLRightInline{Porter C, Reidy PT, Bhattarai N, Sidossis LS, Rasmussen BB. Resistance exercise training alters mitochondrial function in human skeletal muscle. Medicine and science in sports and exercise {[}Internet{]}. 2015;47(9):1922--31. }

\leavevmode\hypertarget{ref-RN2615}{}%
\CSLLeftMargin{8. }
\CSLRightInline{Ørtenblad N, Nielsen J, Boushel R, Söderlund K, Saltin B, Holmberg H-C. The muscle fiber profiles, mitochondrial content, and enzyme activities of the exceptionally well-trained arm and leg muscles of elite cross-country skiers. Frontiers in physiology {[}Internet{]}. 2018;9:1031--1. }

\leavevmode\hypertarget{ref-RN789}{}%
\CSLLeftMargin{10. }
\CSLRightInline{Krieger JW. Single vs. Multiple sets of resistance exercise for muscle hypertrophy: A meta-analysis. J Strength Cond Res {[}Internet{]}. 2010;24(4):1150--9. }

\leavevmode\hypertarget{ref-RN789}{}%
\CSLLeftMargin{10. }
\CSLRightInline{Krieger JW. Single vs. Multiple sets of resistance exercise for muscle hypertrophy: A meta-analysis. J Strength Cond Res {[}Internet{]}. 2010;24(4):1150--9. }

\leavevmode\hypertarget{ref-RN2575}{}%
\CSLLeftMargin{12. }
\CSLRightInline{Evans JW. Periodized resistance training for enhancing skeletal muscle hypertrophy and strength: A mini-review. Frontiers in physiology {[}Internet{]}. 2019;10:13--3. }

\leavevmode\hypertarget{ref-RN2575}{}%
\CSLLeftMargin{12. }
\CSLRightInline{Evans JW. Periodized resistance training for enhancing skeletal muscle hypertrophy and strength: A mini-review. Frontiers in physiology {[}Internet{]}. 2019;10:13--3. }

\leavevmode\hypertarget{ref-RN2572}{}%
\CSLLeftMargin{14. }
\CSLRightInline{Grgic J, Mikulic P, Podnar H, Pedisic Z. Effects of linear and daily undulating periodized resistance training programs on measures of muscle hypertrophy: A systematic review and meta-analysis. PeerJ {[}Internet{]}. 2017;5:e3695--5. }

\leavevmode\hypertarget{ref-RN2572}{}%
\CSLLeftMargin{14. }
\CSLRightInline{Grgic J, Mikulic P, Podnar H, Pedisic Z. Effects of linear and daily undulating periodized resistance training programs on measures of muscle hypertrophy: A systematic review and meta-analysis. PeerJ {[}Internet{]}. 2017;5:e3695--5. }

\leavevmode\hypertarget{ref-RN2571}{}%
\CSLLeftMargin{16. }
\CSLRightInline{Schoenfeld BJ, Ogborn D, Krieger JW. Effects of resistance training frequency on measures of muscle hypertrophy: A systematic review and meta-analysis. Sports Med {[}Internet{]}. 2016;46(11):1689--97. }

\leavevmode\hypertarget{ref-RN2571}{}%
\CSLLeftMargin{16. }
\CSLRightInline{Schoenfeld BJ, Ogborn D, Krieger JW. Effects of resistance training frequency on measures of muscle hypertrophy: A systematic review and meta-analysis. Sports Med {[}Internet{]}. 2016;46(11):1689--97. }

\leavevmode\hypertarget{ref-RN2569}{}%
\CSLLeftMargin{18. }
\CSLRightInline{Schoenfeld BJ, Grgic J, Ogborn D, Krieger JW. Strength and hypertrophy adaptations between low- vs. High-load resistance training: A systematic review and meta-analysis. J Strength Cond Res. 2017;31(12):3508--23. }

\leavevmode\hypertarget{ref-RN2569}{}%
\CSLLeftMargin{18. }
\CSLRightInline{Schoenfeld BJ, Grgic J, Ogborn D, Krieger JW. Strength and hypertrophy adaptations between low- vs. High-load resistance training: A systematic review and meta-analysis. J Strength Cond Res. 2017;31(12):3508--23. }

\leavevmode\hypertarget{ref-RN1612}{}%
\CSLLeftMargin{20. }
\CSLRightInline{Schoenfeld BJ, Ratamess NA, Peterson MD, Contreras B, Sonmez GT, Alvar BA. Effects of different volume-equated resistance training loading strategies on muscular adaptations in well-trained men. J Strength Cond Res {[}Internet{]}. 2014;28(10):2909--18. }

\leavevmode\hypertarget{ref-RN1612}{}%
\CSLLeftMargin{20. }
\CSLRightInline{Schoenfeld BJ, Ratamess NA, Peterson MD, Contreras B, Sonmez GT, Alvar BA. Effects of different volume-equated resistance training loading strategies on muscular adaptations in well-trained men. J Strength Cond Res {[}Internet{]}. 2014;28(10):2909--18. }

\leavevmode\hypertarget{ref-RN2570}{}%
\CSLLeftMargin{22. }
\CSLRightInline{Grgic J, Schoenfeld BJ, Davies TB, Lazinica B, Krieger JW, Pedisic Z. Effect of resistance training frequency on gains in muscular strength: A systematic review and meta-analysis. Sports Med {[}Internet{]}. 2018;48(5):1207--20. }

\leavevmode\hypertarget{ref-RN2570}{}%
\CSLLeftMargin{22. }
\CSLRightInline{Grgic J, Schoenfeld BJ, Davies TB, Lazinica B, Krieger JW, Pedisic Z. Effect of resistance training frequency on gains in muscular strength: A systematic review and meta-analysis. Sports Med {[}Internet{]}. 2018;48(5):1207--20. }

\leavevmode\hypertarget{ref-RN2591}{}%
\CSLLeftMargin{24. }
\CSLRightInline{Nunes JP, Grgic J, Cunha PM, Ribeiro AS, Schoenfeld BJ, Salles BF de, et al. What influence does resistance exercise order have on muscular strength gains and muscle hypertrophy? A systematic review and meta-analysis. Eur J Sport Sci {[}Internet{]}. 2020;1--9. }

\leavevmode\hypertarget{ref-RN2591}{}%
\CSLLeftMargin{24. }
\CSLRightInline{Nunes JP, Grgic J, Cunha PM, Ribeiro AS, Schoenfeld BJ, Salles BF de, et al. What influence does resistance exercise order have on muscular strength gains and muscle hypertrophy? A systematic review and meta-analysis. Eur J Sport Sci {[}Internet{]}. 2020;1--9. }

\leavevmode\hypertarget{ref-RN2492}{}%
\CSLLeftMargin{26. }
\CSLRightInline{Ralston GW, Kilgore L, Wyatt FB, Baker JS. The effect of weekly set volume on strength gain: A meta-analysis. Sports Med {[}Internet{]}. 2017;47(12):2585--601. }

\leavevmode\hypertarget{ref-RN2492}{}%
\CSLLeftMargin{26. }
\CSLRightInline{Ralston GW, Kilgore L, Wyatt FB, Baker JS. The effect of weekly set volume on strength gain: A meta-analysis. Sports Med {[}Internet{]}. 2017;47(12):2585--601. }

\leavevmode\hypertarget{ref-RN793}{}%
\CSLLeftMargin{28. }
\CSLRightInline{Krieger JW. Single versus multiple sets of resistance exercise: A meta-regression. J Strength Cond Res {[}Internet{]}. 2009;23(6):1890--901. }

\leavevmode\hypertarget{ref-RN793}{}%
\CSLLeftMargin{28. }
\CSLRightInline{Krieger JW. Single versus multiple sets of resistance exercise: A meta-regression. J Strength Cond Res {[}Internet{]}. 2009;23(6):1890--901. }

\leavevmode\hypertarget{ref-RN1767}{}%
\CSLLeftMargin{30. }
\CSLRightInline{Schoenfeld BJ, Ogborn D, Krieger JW. Dose-response relationship between weekly resistance training volume and increases in muscle mass: A systematic review and meta-analysis. J Sports Sci {[}Internet{]}. 2016;1--0. }

\leavevmode\hypertarget{ref-RN1767}{}%
\CSLLeftMargin{30. }
\CSLRightInline{Schoenfeld BJ, Ogborn D, Krieger JW. Dose-response relationship between weekly resistance training volume and increases in muscle mass: A systematic review and meta-analysis. J Sports Sci {[}Internet{]}. 2016;1--0. }

\leavevmode\hypertarget{ref-RN2130}{}%
\CSLLeftMargin{32. }
\CSLRightInline{Baz-Valle E, Fontes-Villalba M, Santos-Concejero J. Total number of sets as a training volume quantification method for muscle hypertrophy: A systematic review. J Strength Cond Res. 2018; }

\leavevmode\hypertarget{ref-RN2130}{}%
\CSLLeftMargin{32. }
\CSLRightInline{Baz-Valle E, Fontes-Villalba M, Santos-Concejero J. Total number of sets as a training volume quantification method for muscle hypertrophy: A systematic review. J Strength Cond Res. 2018; }

\leavevmode\hypertarget{ref-RN1476}{}%
\CSLLeftMargin{34. }
\CSLRightInline{Berger R. Effect of varied weight training programs on strength. Research Quarterly American Association for Health, Physical Education and Recreation {[}Internet{]}. 1962;33(2):168--81. }

\leavevmode\hypertarget{ref-RN1476}{}%
\CSLLeftMargin{34. }
\CSLRightInline{Berger R. Effect of varied weight training programs on strength. Research Quarterly American Association for Health, Physical Education and Recreation {[}Internet{]}. 1962;33(2):168--81. }

\leavevmode\hypertarget{ref-RN794}{}%
\CSLLeftMargin{36. }
\CSLRightInline{Carpinelli RN, Otto RM. Strength training. Single versus multiple sets. Sports Med {[}Internet{]}. 1998;26(2):73--84. }

\leavevmode\hypertarget{ref-RN794}{}%
\CSLLeftMargin{36. }
\CSLRightInline{Carpinelli RN, Otto RM. Strength training. Single versus multiple sets. Sports Med {[}Internet{]}. 1998;26(2):73--84. }

\leavevmode\hypertarget{ref-RN2538}{}%
\CSLLeftMargin{38. }
\CSLRightInline{Feigenbaum MS, Pollock ML. Prescription of resistance training for health and disease. Med Sci Sports Exerc. 1999;31(1):38--45. }

\leavevmode\hypertarget{ref-RN2538}{}%
\CSLLeftMargin{38. }
\CSLRightInline{Feigenbaum MS, Pollock ML. Prescription of resistance training for health and disease. Med Sci Sports Exerc. 1999;31(1):38--45. }

\leavevmode\hypertarget{ref-RN2568}{}%
\CSLLeftMargin{39. }
\CSLRightInline{Junyoung H, Corinna NR, John DS, Sukho L. Low volume progressive single set of resistance training is as effective as high volume multiple sets of resistance protocol on muscle strength and power. International journal of applied sports sciences : IJASS. 2015;27(1):33--42. }

\leavevmode\hypertarget{ref-RN2201}{}%
\CSLLeftMargin{41. }
\CSLRightInline{Carpinelli RN. Science versus opinion. British journal of sports medicine {[}Internet{]}. 2004;38(2):240--2. }

\leavevmode\hypertarget{ref-RN2201}{}%
\CSLLeftMargin{41. }
\CSLRightInline{Carpinelli RN. Science versus opinion. British journal of sports medicine {[}Internet{]}. 2004;38(2):240--2. }

\leavevmode\hypertarget{ref-RN2465}{}%
\CSLLeftMargin{43. }
\CSLRightInline{Ribeiro AS, Schoenfeld BJ, Pina FLC, Souza MF, Nascimento MA, Santos L dos, et al. Resistance training in older women: Comparison of single vs. Multiple sets on muscle strength and body composition. Isokinetics and Exercise Science. 2015;23:53--60. }

\leavevmode\hypertarget{ref-RN2465}{}%
\CSLLeftMargin{43. }
\CSLRightInline{Ribeiro AS, Schoenfeld BJ, Pina FLC, Souza MF, Nascimento MA, Santos L dos, et al. Resistance training in older women: Comparison of single vs. Multiple sets on muscle strength and body composition. Isokinetics and Exercise Science. 2015;23:53--60. }

\leavevmode\hypertarget{ref-RN2464}{}%
\CSLLeftMargin{45. }
\CSLRightInline{Correa CS, Teixeira BC, Cobos RC, Macedo RC, Kruger RL, Carteri RB, et al. High-volume resistance training reduces postprandial lipaemia in postmenopausal women. J Sports Sci. 2015;33(18):1890--901. }

\leavevmode\hypertarget{ref-RN2464}{}%
\CSLLeftMargin{45. }
\CSLRightInline{Correa CS, Teixeira BC, Cobos RC, Macedo RC, Kruger RL, Carteri RB, et al. High-volume resistance training reduces postprandial lipaemia in postmenopausal women. J Sports Sci. 2015;33(18):1890--901. }

\leavevmode\hypertarget{ref-RN2463}{}%
\CSLLeftMargin{47. }
\CSLRightInline{Bottaro M, Veloso J, Wagner D, Gentil P. Resistance training for strength and muscle thickness: Effect of number of sets and muscle group trained. Science \& Sports {[}Internet{]}. 2011;26(5):259--64. }

\leavevmode\hypertarget{ref-RN2463}{}%
\CSLLeftMargin{47. }
\CSLRightInline{Bottaro M, Veloso J, Wagner D, Gentil P. Resistance training for strength and muscle thickness: Effect of number of sets and muscle group trained. Science \& Sports {[}Internet{]}. 2011;26(5):259--64. }

\leavevmode\hypertarget{ref-RN1570}{}%
\CSLLeftMargin{49. }
\CSLRightInline{Radaelli R, Fleck SJ, Leite T, Leite RD, Pinto RS, Fernandes L, et al. Dose response of 1, 3 and 5 sets of resistance exercise on strength, local muscular endurance and hypertrophy. J Strength Cond Res {[}Internet{]}. 2014; }

\leavevmode\hypertarget{ref-RN1570}{}%
\CSLLeftMargin{49. }
\CSLRightInline{Radaelli R, Fleck SJ, Leite T, Leite RD, Pinto RS, Fernandes L, et al. Dose response of 1, 3 and 5 sets of resistance exercise on strength, local muscular endurance and hypertrophy. J Strength Cond Res {[}Internet{]}. 2014; }

\leavevmode\hypertarget{ref-RN1518}{}%
\CSLLeftMargin{51. }
\CSLRightInline{Radaelli R, Wilhelm EN, Botton CE, Rech A, Bottaro M, Brown LE, et al. Effects of single vs. Multiple-set short-term strength training in elderly women. Age (Dordr) {[}Internet{]}. 2014;36(6):9720. }

\leavevmode\hypertarget{ref-RN1518}{}%
\CSLLeftMargin{51. }
\CSLRightInline{Radaelli R, Wilhelm EN, Botton CE, Rech A, Bottaro M, Brown LE, et al. Effects of single vs. Multiple-set short-term strength training in elderly women. Age (Dordr) {[}Internet{]}. 2014;36(6):9720. }

\leavevmode\hypertarget{ref-RN1474}{}%
\CSLLeftMargin{53. }
\CSLRightInline{McBride JM, Blaak JB, Triplett-McBride T. Effect of resistance exercise volume and complexity on EMG, strength, and regional body composition. Eur J Appl Physiol {[}Internet{]}. 2003;90(5-6):626--32. }

\leavevmode\hypertarget{ref-RN1474}{}%
\CSLLeftMargin{53. }
\CSLRightInline{McBride JM, Blaak JB, Triplett-McBride T. Effect of resistance exercise volume and complexity on EMG, strength, and regional body composition. Eur J Appl Physiol {[}Internet{]}. 2003;90(5-6):626--32. }

\leavevmode\hypertarget{ref-RN1472}{}%
\CSLLeftMargin{55. }
\CSLRightInline{Galvao DA, Taaffe DR. Resistance exercise dosage in older adults: Single- versus multiset effects on physical performance and body composition. J Am Geriatr Soc {[}Internet{]}. 2005;53(12):2090--7. }

\leavevmode\hypertarget{ref-RN1472}{}%
\CSLLeftMargin{55. }
\CSLRightInline{Galvao DA, Taaffe DR. Resistance exercise dosage in older adults: Single- versus multiset effects on physical performance and body composition. J Am Geriatr Soc {[}Internet{]}. 2005;53(12):2090--7. }

\leavevmode\hypertarget{ref-RN1456}{}%
\CSLLeftMargin{57. }
\CSLRightInline{Starkey DB, Pollock ML, Ishida Y, Welsch MA, Brechue WF, Graves JE, et al. Effect of resistance training volume on strength and muscle thickness. Med Sci Sports Exerc {[}Internet{]}. 1996;28(10):1311--20. }

\leavevmode\hypertarget{ref-RN1456}{}%
\CSLLeftMargin{57. }
\CSLRightInline{Starkey DB, Pollock ML, Ishida Y, Welsch MA, Brechue WF, Graves JE, et al. Effect of resistance training volume on strength and muscle thickness. Med Sci Sports Exerc {[}Internet{]}. 1996;28(10):1311--20. }

\leavevmode\hypertarget{ref-RN1454}{}%
\CSLLeftMargin{59. }
\CSLRightInline{Ostrowski KJ, Wilson GJ, Weatherby R, Murphy PW, Lyttle AD. The effect of weight training volume on hormonal output and muscular size and function. Journal of Strength and Conditioning Research {[}Internet{]}. 1997;11(3):148--54. }

\leavevmode\hypertarget{ref-RN1454}{}%
\CSLLeftMargin{59. }
\CSLRightInline{Ostrowski KJ, Wilson GJ, Weatherby R, Murphy PW, Lyttle AD. The effect of weight training volume on hormonal output and muscular size and function. Journal of Strength and Conditioning Research {[}Internet{]}. 1997;11(3):148--54. }

\leavevmode\hypertarget{ref-RN1384}{}%
\CSLLeftMargin{61. }
\CSLRightInline{Rhea MR, Alvar BA, Ball SD, Burkett LN. Three sets of weight training superior to 1 set with equal intensity for eliciting strength. J Strength Cond Res {[}Internet{]}. 2002;16(4):525--9. }

\leavevmode\hypertarget{ref-RN1384}{}%
\CSLLeftMargin{61. }
\CSLRightInline{Rhea MR, Alvar BA, Ball SD, Burkett LN. Three sets of weight training superior to 1 set with equal intensity for eliciting strength. J Strength Cond Res {[}Internet{]}. 2002;16(4):525--9. }

\leavevmode\hypertarget{ref-RN1382}{}%
\CSLLeftMargin{63. }
\CSLRightInline{Cannon J, Marino FE. Early-phase neuromuscular adaptations to high- and low-volume resistance training in untrained young and older women. J Sports Sci {[}Internet{]}. 2010;28(14):1505--14. }

\leavevmode\hypertarget{ref-RN1382}{}%
\CSLLeftMargin{63. }
\CSLRightInline{Cannon J, Marino FE. Early-phase neuromuscular adaptations to high- and low-volume resistance training in untrained young and older women. J Sports Sci {[}Internet{]}. 2010;28(14):1505--14. }

\leavevmode\hypertarget{ref-RN776}{}%
\CSLLeftMargin{65. }
\CSLRightInline{Ronnestad BR, Egeland W, Kvamme NH, Refsnes PE, Kadi F, Raastad T. Dissimilar effects of one- and three-set strength training on strength and muscle mass gains in upper and lower body in untrained subjects. J Strength Cond Res {[}Internet{]}. 2007;21(1):157--63. }

\leavevmode\hypertarget{ref-RN776}{}%
\CSLLeftMargin{65. }
\CSLRightInline{Ronnestad BR, Egeland W, Kvamme NH, Refsnes PE, Kadi F, Raastad T. Dissimilar effects of one- and three-set strength training on strength and muscle mass gains in upper and lower body in untrained subjects. J Strength Cond Res {[}Internet{]}. 2007;21(1):157--63. }

\leavevmode\hypertarget{ref-RN834}{}%
\CSLLeftMargin{67. }
\CSLRightInline{Mitchell CJ, Churchward-Venne TA, West DW, Burd NA, Breen L, Baker SK, et al. Resistance exercise load does not determine training-mediated hypertrophic gains in young men. J Appl Physiol (1985) {[}Internet{]}. 2012;113(1):71--7. }

\leavevmode\hypertarget{ref-RN834}{}%
\CSLLeftMargin{67. }
\CSLRightInline{Mitchell CJ, Churchward-Venne TA, West DW, Burd NA, Breen L, Baker SK, et al. Resistance exercise load does not determine training-mediated hypertrophic gains in young men. J Appl Physiol (1985) {[}Internet{]}. 2012;113(1):71--7. }

\leavevmode\hypertarget{ref-RN1607}{}%
\CSLLeftMargin{69. }
\CSLRightInline{Sooneste H, Tanimoto M, Kakigi R, Saga N, Katamoto S. Effects of training volume on strength and hypertrophy in young men. J Strength Cond Res {[}Internet{]}. 2013;27(1):8--13. }

\leavevmode\hypertarget{ref-RN1607}{}%
\CSLLeftMargin{69. }
\CSLRightInline{Sooneste H, Tanimoto M, Kakigi R, Saga N, Katamoto S. Effects of training volume on strength and hypertrophy in young men. J Strength Cond Res {[}Internet{]}. 2013;27(1):8--13. }

\leavevmode\hypertarget{ref-RN2563}{}%
\CSLLeftMargin{71. }
\CSLRightInline{Lin CH, Platt MD, Ficarro SB, Hoofnagle MH, Shabanowitz J, Comai L, et al. Mass spectrometric identification of phosphorylation sites of rRNA transcription factor upstream binding factor. Am J Physiol Cell Physiol {[}Internet{]}. 2007;292(5):C1617--24. }

\leavevmode\hypertarget{ref-RN2563}{}%
\CSLLeftMargin{71. }
\CSLRightInline{Lin CH, Platt MD, Ficarro SB, Hoofnagle MH, Shabanowitz J, Comai L, et al. Mass spectrometric identification of phosphorylation sites of rRNA transcription factor upstream binding factor. Am J Physiol Cell Physiol {[}Internet{]}. 2007;292(5):C1617--24. }

\leavevmode\hypertarget{ref-RN2602}{}%
\CSLLeftMargin{72. }
\CSLRightInline{Voit R, Hoffmann M, Grummt I. Phosphorylation by G1-specific cdk-cyclin complexes activates the nucleolar transcription factor UBF. Embo j {[}Internet{]}. 1999;18(7):1891--9. }

\leavevmode\hypertarget{ref-RN2604}{}%
\CSLLeftMargin{73. }
\CSLRightInline{Stefanovsky VY, Pelletier G, Hannan R, Gagnon-Kugler T, Rothblum LI, Moss T. An immediate response of ribosomal transcription to growth factor stimulation in mammals is mediated by ERK phosphorylation of UBF. Molecular Cell {[}Internet{]}. 2001;8(5):1063--73. }

\leavevmode\hypertarget{ref-RN2618}{}%
\CSLLeftMargin{74. }
\CSLRightInline{Newlands S, Levitt LK, Robinson CS, Karpf AB, Hodgson VR, Wade RP, et al. Transcription occurs in pulses in muscle fibers. Genes \& development {[}Internet{]}. 1998;12(17):2748--58. }

\leavevmode\hypertarget{ref-RN2616}{}%
\CSLLeftMargin{75. }
\CSLRightInline{Kirby TJ, Patel RM, McClintock TS, Dupont-Versteegden EE, Peterson CA, McCarthy JJ. Myonuclear transcription is responsive to mechanical load and DNA content but uncoupled from cell size during hypertrophy. Molecular biology of the cell {[}Internet{]}. 2016;27(5):788--98. }

\end{CSLReferences}

% Index?

\end{document}
